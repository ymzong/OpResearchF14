\documentclass{article} % For LaTeX2e
\usepackage{nips13submit_e,times}
\usepackage[toc,page]{appendix}
\usepackage{listings}
%\usepackage{hyperref}
\usepackage[hyphens]{url}
\usepackage{bbm}
\usepackage{amsfonts, amssymb}
\usepackage{alltt}
\usepackage{multirow, caption}
\usepackage{algorithm}
\usepackage{amsmath}
\usepackage{graphicx}
\usepackage[noend]{algpseudocode}
\usepackage{etoolbox}
\usepackage{hyperref}
\usepackage{fancyvrb}
\usepackage{tgcursor}
\usepackage{longtable}
\newcommand\userinput[1]{\textbf{#1}}
\newcommand\comment[1]{\textit{#1}}
\newcommand\stdout[1]{\textsl{#1}}

\makeatletter
\preto{\@verbatim}{\topsep=0pt \partopsep=0pt }
\makeatother

% Tweak algorithmic package.
\renewcommand{\algorithmicrequire}{\textbf{Input:}}
\renewcommand{\algorithmicensure}{\textbf{Output:}}

\title{Freshman Seminar Assignment Problem\\Final Report}

\author{
Keenan Gao \\
\And
Binghui Ouyang \\
\And
Hanwen Zhang \\
\And
Yiming Zong\\ \\
\and Department of Mathematical Sciences \\ Carnegie Mellon University \\ Pittsburgh, PA 15213
}

\newcommand{\fix}{\marginpar{FIX}}
\newcommand{\new}{\marginpar{NEW}}

\nipsfinalcopy % Uncomment for camera-ready version

\begin{document}

% Title
\maketitle

% Paper Abstract
\begin{abstract}
    \par\qquad As the number of college students increases, an optimization algorithm that can automatically assign students to classes becomes a pressing need in many universities. In attempts of solving the problem, various questions arose. What is the objective function to optimize? What aspects need to be considered? In this final report, we will deliver solutions to the Freshman Seminar Assignment Problem, our team's final project for Operations Research II, Fall 2014. We will first briefly describe the real-world problem and create a basic mathematical model based on the data. Then, we will feed the pre-processed data into several different algorithms with various objective functions and run-time constraints, and compare their final results. Eventually, we will discuss some generalizations of our current problem and propose potential algorithms for solving them. Our algorithms improve the quality of assignment significantly in terms of student preference compared to the traditional approach of manually assigning students.
\end{abstract}
\vspace{\fill}
\pagebreak

%
% Main Body Starts
%
\section{Problem Overview}
   \par\qquad The assignment problem we are solving in this project was initially broached by Dietrich College of Humanities and Social Sciences at Carnegie Mellon University. The main goal is to assign freshmen to mandatory seminars in a way such that all seminars are filled, and that students are assigned to the seminars that they are interested in. In the real-world dataset we are provided with, there are $308$ students and $22$ seminars. Each seminar can have at most 16 students, and each student must enroll in exactly one seminar.
   \par\qquad To facilitate the matching process, students are asked to rank one ``first-choice'' seminar and three ``second-choice'' seminars. We associate costs with assigning a student to a seminar based on the student's preference for the seminar. To find the optimal assignment, we approached the problem in two ways -- minimizing the total cost across all seminars and smoothing out enrolled students' preference for each seminar.

\section{Mathematical Model}
\par\qquad Based on the problem description in \emph{Section 1}, we can build a mathematical model for it in order to describe an algorithm for solving the problem with any valid input data.

\subsection{General Input}
\begin{itemize}
    \item $n$: Number of students ($n > 0$);
    \item $m$: Number of seminars ($m > 0$);
    \item $k$: Maximum number of selections that a student can make ($1 \leq k \leq m$);
    \item $s_{i,j}$: The $j^{\text{th}}$ selection of $i^{\text{th}}$ student, where $1\leq i \leq n$ and $1\leq j \leq k$. $s_{i,j}=0$ when the Student $i$ makes no corresponding choice for Rank $j$;
    \item $q_k$: The quota for $k^{\text{th}}$ seminar, where $1\leq k \leq m$.
\end{itemize}

\subsection{Input Constraints}
\begin{itemize}
    \item Positivity: $n,m,k>0$, $\forall k\in\{1,\cdots,m\}, q_k>0$;
    \item Number of selections for student is bounded by number of available seminars: $k \leq m$;
    \item Student rankings are valid and unique: $\forall (i,j), 1\leq s_{i,j} \leq m$. And, for each $i$, all non-zero entries $s_{i,j}$'s take unique values.
\end{itemize}

\subsection{Decision Variables}
\begin{itemize}
    \item $Y_{i,j}$: Indicator variables for whether Student $i$ is assigned to Seminar $j$, where $1\leq i \leq n$ and $1\leq j \leq k$;
\end{itemize}

\subsection{Data Pre-Processing}
    \par\qquad In order to deal with cases when a student is only willing or allowed to rank $k'<m$ seminars, we automatically set all ``unassigned'' priorities to a large value $\textit{M}$. Also, we change the representation of students' preference from \emph{(student, ranking) $\mapsto$ seminar} to \emph{(student, seminar) $\mapsto$ ranking} to make further calculations easier, i.e.6
        $$X_{i,j}:=\begin{cases}
                     l &\text{if Student $i$ ranked $j$ as $l^{\text{th}}$ option, or $s_{i,l}=j$ for some $l\in\{1,\cdots,k\}$}\\
                     \textit{M}         &\text{if Seminar $j$ is not on Student $i$'s list, or $s_{i,l} \neq j$ for all $l\in\{1,\cdots,k\}$}
                   \end{cases},$$
    where $\textit{M}$ is an arbitrarily large value in order to discourage the algorithm from assigning a student to a seminar that s/he did not list.

\subsection{General Constraints}
    \begin{itemize}
    \item $Y_{i,j}$'s are indeed indicator variables: $\forall (i,j)\in[n]\times[k], Y_{i,j}\in\{0,1\}$;
    \item Each student is assigned to precisely one seminar: $\forall i\in[n]$, $\sum_{l=1}^{m}{Y_{i,l}}=1$;
    \item Each seminar is within enrollment quota: $\forall j\in[m]$, $\sum_{l=1}^{n}{Y_{l,j}} \leq q_j$;
\end{itemize}

%
% Heuristic Selection
%
\section{Approach for Various Heuristic Functions}
    \par\qquad Due to the flexibility of the original problem, we are proposing different objective functions for optimization, including minimizing the total ``ranks'' given by the students, and minimizing the variance of student preference across different seminars. In the following sub-sections we briefly introduce our approach for each heuristic.

\subsection{Minimize Total Rank of Students}
    \par\qquad In this case, our goal is to minimize the sum of all student rankings for their assigned seminars. To do so, our objective is to minimize $W = \sum_{i=1}^{n}{\sum_{j=1}^{m}{X_{i,j}Y_{i,j}}}$. This would guarantee that students are as satisfied as they can be \emph{overall}.

\subsection{Minimize Variance of Students Preference across Seminars}
    \par\qquad In addition to minimizing the total rank of students, it would also be helpful if we could ``balance out'' students' preference of their assigned seminar for each seminar. For example, we do not want to have an assignment where some seminar has all students listing it as their first choice, yet some other seminar has none of the students listing it in their choices at all. In order to approximate the optimal solution, for each seminar we may enforce a hard limit on the number of students enrolled that ranked it as first tier, second tier, etc.
    \par\qquad Meanwhile, the hard limits can clearly interfere with the optimal solution that minimizes the total rank of students in the final assignment. Therefore, care needs to be taken when picking the hard limits, such that there is a balance between maximizing overall satisfaction and minimizing variance of satisfaction across seminars.

%
% Optimization Algorithms
%
% Exact Algorithm (Hungarian)
%
\section{Exact Algorithm for Minimizing Total Rank}
    \par\qquad If we only consider the objective function in \emph{Section 3.1}, the problem can be reduced to an \emph{Assignment Problem} when we include ``dummy seminars'' and ``dummy students''. After applying \emph{Hungarian Algorithm}, we can obtain an absolute optimal solution that minimizes the total ``cost'' of students. Details about the algorithm are as follows:
    
\subsection{Data Pre-Processing}
    \par\qquad We start with the matrix $X_{i,j}$ as obtained in \emph{Section 2.4}. For each column that represents Seminar $j$, we create extra $(j-1)$ dummy seminars by duplicating the same column $q_j$ times. After that, we make our cost matrix square by adding zero rows (i.e. dummy students) at the bottom of the cost matrix. This gives us a matrix that we may feed into \emph{Hungarian Algorithm}.

\subsection{Hungarian Algorithm}
    \par\qquad Given the square cost matrix from previous section, \emph{Hungarian Algorithm} returns a \emph{student-dummy seminar} assignment with minimal total cost in polynomial time \cite{mt90}. From the result, we may simply assign each student to the actual seminar that the dummy seminar in the augmented cost matrix corresponds to.

%
% Approximation Algorithm (Greedy)
%
\section{Approximation Algorithm for Constraining Variance}
    \par\qquad While the previous exact algorithm minimizes the sum of ranks of students across all seminars, it does \emph{not} balance out the preference of students in different seminars. In order to do so, a heuristic for approximation is to place an artificial hard quota, $\mathcal{Q}\in(0,1]$, on the proportion of students in any seminar that place it as their first option. While the exact optimal solution for different values of $\mathcal{Q}$ can be obtained by running Hungarian algorithm with different combinations of dummy seminars, its runtime (more than one hour for $n=300$) makes it undesirable.
    \par\qquad In this case, our solution is to use a randomized balanced algorithm with user-defined number of iterations, and then return the best solution to the user. This algorithm allows us to determine a desired distribution of student interest and minimize its ensuing variance. Following is the detail of the algorithm:

\subsection{Balanced Algorithm}
    \par\qquad For the balanced algorithm, we fix a value of $\mathcal{Q}$. For each seminar $j$, we randomly select up to $\mathcal{Q}\cdot q_j$ students who have selected it as their first choice, and assign them to that seminar. If the number of students who have ranked it first is smaller than $\mathcal{Q}\cdot q_j$, then all of those students are assigned to that seminar. After each seminar has been filled with up to $\mathcal{Q}\cdot q_j$ first choice students, we randomly fill the remaining $(q_j - \mathcal{Q}\cdot q_j)$ seats in the seminar with the students who have ranked that seminar as their second choice. If there are students remaining after this second round of assignments, we then randomly assign them each a seminar.
    \par\qquad Due to the random nature of the algorithm, it is best to run it multiple times in order to select an assignment that gives minimum total cost. Each iteration of the algorithm is outlined as follows:
    \begin{algorithm}
        \caption{Balanced Algorithm}
        \begin{algorithmic}
            \State{UnassignedStudent $\gets \{1,2,\cdots,n\}$}
            \State{$\text{Roster}_j \gets \varnothing$ for each $j\in[m]$}
            \For{j = 1 to m}
                \State{FirstChoiceSet$_j$} $\gets$ \text{RandomSubset}$\left(\{i\in[n] \mid X_{i,j}=1 \} \cap \text{UnassignedStudent}, \mathcal{Q}\cdot q_j\right)$
                \State{Roster$_j$} $\gets$ Roster$_j\;\cup$ FirstChoiceSet$_j$
                \State{UnassignedStudent} $\gets$ UnassignedStudent $\setminus$ FirstChoiceSet$_j$
            \EndFor
            \For{j = 1 to m}
                \State{SecondChoiceSet$_j$} $\gets$ \text{RandomSubset}$\left(\{i\in[n] \mid X_{i,j}=1 \} \cap \text{UnassignedStudent}, q_j - |\text{Roster$_j$}|\right)$
                \State{Roster$_j$} $\gets$ Roster$_j\;\cup$ SecondChoiceSet$_j$
                \State{UnassignedStudent} $\gets$ UnassignedStudent $\setminus$ SecondChoiceSet$_j$
            \EndFor
            \For{s {\bf in} UnassignedStudent}
                \State Assign $s$ to a random available seminar
            \EndFor
            \Ensure{$\text{Roster}_j \gets \text{list of students in Seminar $j$ based on balanced algorithm}$ for each $j\in[m]$}
        \end{algorithmic}
    \end{algorithm}

%
% Summary
%
\section{Summary of Results}
    \par\qquad So far we have implemented the two algorithms above to solve the seminar assignment problem, namely the Exact Algorithm which applies the Hungarian Algorithm, and the Approximation Algorithm which utilizes randomization. Both algorithms are tested on the real data for the incoming class of Dietrich College for Year 2013 $(n=308, m=22)$. The cost for assigning a student to a seminar is defined as $g(x)=2x^2$, where $x$ is the tier that student ranks the seminar; if the student does not rank the seminar, the cost would be $g(x)=10,000$, which is a huge constant inspired by Big-M method in linear programming.

\subsection{Exact Algorithm}
\par\qquad We use the Hungarian Algorithm to find the student-seminar assignment with the minimal total cost. The algorithm gives a very satisfactory assignment within five minutes. To be specific, among all the 308 students, 207 of them are assigned to their first choice of seminars, consisting of 67.2\% of the population; and 82 are assigned to their second choice of classes, consisting of 26.6\% of the population. Therefore, a total of 93.8\% of the students are assigned to either their first or second choice of seminars. The remaining 19 students do not provide any preference of seminars, so they are randomly assigned to seminars which are not full. We also get a satisfying balance among seminar enrollments. Among all the 22 existing seminars, 18 of them get full enrollment of 16 students (16 was the highest enrollment allowed in 2013); one of the seminar gets 15 students; one gets 5 students; and the other two seminars left get 0 students enrolled. For the seminars with zero enrollment, we would simply cancel it. The total cost of the optimal solution is $1,900,164$.

\subsection{Approximation Algorithm}
\par\qquad The Approximation Algorithm returns a quite satisfying result with 10,000 iterations in under two and a half minutes. Among the 308 freshmen students, 151 of them are assigned to their first choices, which consists of 49.0\% of the total population of students; 132 are assigned to their second choices, which consists of 42.9\% of the population; 5 students are assigned to their later choices; and 19 of them do not show any preference of seminars and are assigned to a random available seminar. Therefore, a total of 91.9\% of the students who indicated preferences are assigned to either their first or second choice. Regarding the balance of enrollments among the seminars, 15 of the seminars get full enrollment of 16 students; 3 of the seminars get pretty good enrollments of 15, 14 and 13 students respectively; and 3 of the seminars are poorly enrolled with 7, 5 and 2 students respectively. The total cost of the optional solution is $2,200,270$.

\subsection{Comparison \& Analysis}

\par\qquad According to the results of the Exact Algorithm and the Approximation Algorithm, both are able to provide us with quite satisfying assignments. More than 90\% of the students are able to get in their first or second choices of seminars. In addition, the class enrollments are balanced. 

\par\qquad However, both algorithms have their own pros and cons. The Approximation Algorithm has a faster and scalable performance compared to the Exact Algorithm because it uses randomization instead of calculating the best result. Therefore, when the dataset becomes very large, it is better to choose the Approximation Algorithm. However, the relative performance of the Approximation Algorithm cannot be guaranteed compared to the Exact Algorithm, and it is absolutely not as good as the Exact Algorithm which always gives the minimum total cost. Therefore, when the dataset is not huge and the runtime from using the Exact Algorithm is realistic, the Exact Algorithm should be a better choice as it provides us with the best solution.

\subsection{Limitations}
    \par\qquad As mentioned above, despite the satisfactory performance of the exact algorithm, its run-time complexity $\mathcal{O}(n^3)$ makes the algorithm undesirable for $n>1000$. Therefore, further work should be focused on more effective heuristics for the approximation algorithm in order to improve the quality of result under a reasonable time constraint.

\section{Further Work \& Enhancements}

\subsection{Allowing flexible input parameters}
    \par\qquad For our implementation, the main test case is for seminars with the same enrollment quota, and that each student makes one ``first-choice'' and three ``second-choices''. However, our implementation of the algorithms in \emph{Sections 4-5} also supports seminars with various quotas and also students that rank arbitrary number of seminars in arbitrary number of tiers. This makes our implementation applicable to much more real-world cases because in general the course sizes need not be the same, and students should be given the opportunity to make flexible selections.

\subsection{Supporting bi-directional preference/cost parameters with Stable Marriage Algorithm}
   \par\qquad One of the features we could implement in the future might be that the final assignment does not only depends on the students' rankings on seminars but also the seminars' rankings of students. For example, if a certain seminar is very major-oriented and prefers history major students, then we will want to enhance our algorithm by accommodating for these bi-directional preferences. Algorithms like the Stable Marriage Algorithm would be useful to reference, or we may simply use the sum of cost in both students' and seminars' perspectives as parameters in the algorithms specified in \emph{Sections 4-5}.
   
\subsection{Stable Assignment Optimization}
    \par\qquad Similar to the principle of \emph{Stable Marriage Problem}, in the final seminar assignment we do not want to have two students \emph{A} and \emph{B}, such that \emph{A} prefers \emph{B}'s section, and also vice versa (we call those two students \emph{rogue pair}). This can be done by scanning each pair of students and fixing every \emph{rogue pair}. The algorithm is outlined as follows:
    \begin{algorithm}
        \caption{Rogue-Pair Fixing Algorithm}
        \begin{algorithmic}
        \Require $\text{asgn}_i \gets \text{current seminar assignment for Student $i$}$
        \Ensure (i,j) if we found a rogue pair, otherwise null
        \Function{FindRoguePair}{\text{asgn}}
            \For {i=1 to n}
                \For {j=i+1 to n}
                    \If{Student $i$ and $j$ prefer each other's seminar}
                        \Return{(i,j)}
                    \EndIf
                \EndFor
            \EndFor
            \Return{null}
        \EndFunction
        
        \null
        
        \Require $\text{asgn}_i \gets \text{seminar assignment for Student $i$ based on greedy algorithm}$
        \Ensure $\text{asgn}_i$: rogue pair-free assignment for Student $i$
                \State p $\gets$ FindRoguePair(asgn)
                \While {p {\bf not} null}
                    \State (i,j) $\gets$ p
                    \State $\text{asgn}_i \leftrightarrow \text{asgn}_j$
                    \State p $\gets$ FindRoguePair(asgn)
                \EndWhile
        
        \end{algorithmic}
    \end{algorithm}
    
\subsection{Parallelize Approximation Algorithm}
    \par\qquad Since the approximation algorithm in \emph{Section Five} depends on a user-defined number of independent trials, we can actually parallelize the algorithm by using multiple threads, such that each worker is able to make trials and aggregate the result to the \emph{Master} node. For a modern machine with multiple CPU cores, this is able to reduce the algorithm runtime by at least 50\%.

\vskip .3in
\section*{Acknowledgements}
\par\qquad The authors would like to thank Professor Alan Frieze for proposing the Hungarian algorithm in \emph{Section 5} and for holding weekly meetings to track our progress. Also, we thank Professor Brian W. Junker, Professor Joseph E. Devine, and Gloria P. Hill for providing us with the real seminar assignment data for Year 2013. Finally, the thanks goes to Brian Clapper for his Python implementation of Hungarian algorithm, and to Eric Wood, who implemented the excel-to-\LaTeX\ utility.

\urlstyle{rm}
\vskip .2in
\begin{thebibliography}{9}
\bibitem{mt90}
Martello, Silvano, and Paolo Toth. Knapsack Problems: Algorithms and Computer Implementations. Chichester: J. Wiley \& Sons, 1990. Print.
%\url{http://www.or.deis.unibo.it/kp/Chapter7.pdf}
\end{thebibliography}

\vspace{\fill}
\pagebreak
\section*{Appendices}

\subsection*{Appendix A: Raw Data And Results for Seminar Assignment Problem}
%\captionof{table}{Fall 2014 Dietrich College Freshman Seminar interests and Freshman Seminar assigned}
\hoffset=-.8in
\begin{longtable}{| l | l | l | l | l |}
\hline
        {\bf Student ID:} & \begin{tabular}{@{}c@{}}{\bf Student Selection:}\\ {\small Identify 1st choice and 3 2nd choices}\\(ex., 7:  11, 15, 2)\end{tabular}& \begin{tabular}{@{}c@{}}{\bf Assigned by Hand:}\\ {\footnotesize Notes: if blank, none assigned}\end{tabular} & \begin{tabular}{@{}c@{}}{\bf Assigned by}\\{\bf Exact Algorithm:}\end{tabular} & \begin{tabular}{@{}c@{}}{\bf Assigned by}\\{\bf Approx Algorithm:}\end{tabular} \\ \hline
        1 & 8: 14, 20, 3 &  & 3 & 3 \\ \hline
        2 & 14: 20, 12, 2 &  & 14 & 14 \\ \hline
        3 & 17: 13, 2 &  & 17 & 17 \\ \hline
        4 & 14: 6, 19, 13 &  & 13 & 13 \\ \hline
        5 & 12: 9, 20, 14 &  & 12 & 20 \\ \hline
        6 & 12: 20, 6, 3 &  & 12 & 12 \\ \hline
        7 & 14 : 2, 12, 11 &  & 14 & 14 \\ \hline
        8 & 12: 20, 6, 7 &  & 12 & 20 \\ \hline
        9 & 2: 10, 6, 3 &  & 3 & 3 \\ \hline
        10 & 2: 7, 6, 4 &  & 2 & 7 \\ \hline
        11 & 1 & 1 & 1 & 1 \\ \hline
        12 & 1 & 1 & 1 & 1 \\ \hline
        13 & 1 & 1 & 1 & 1 \\ \hline
        14 & 1: 15, 6, 13 & 1 & 1 & 1 \\ \hline
        15 & 1: 8 & 1 & 1 & 1 \\ \hline
        16 & 1 & 1 & 1 & 1 \\ \hline
        17 &  & 1 & 16 & 5 \\ \hline
        18 & 15: 8, 1, 9 & 1 & 15 & 15 \\ \hline
        19 & 1: 16, 15, 21 & 1 & 1 & 1 \\ \hline
        20 & 1: 17, 16, 10 & 1 & 1 & 16 \\ \hline
        21 & 1: 4, 18, 7 & 1 & 1 & 1 \\ \hline
        22 & 20: 18, 1, 12 & 1 & 20 & 20 \\ \hline
        23 & 1: 2, 9, 8 & 1 & 1 & 1 \\ \hline
        24 & 1 & 1 & 1 & 1 \\ \hline
        25 &  & 2 & 16 & 13 \\ \hline
        26 & 18: 14,13,9 & 2 & 18 & 18 \\ \hline
        27 & 2: 14 & 2 & 2 & 2 \\ \hline
        28 & 14: 7, 2, 12 & 2 & 14 & 2 \\ \hline
        29 & 14: 6, 2, 12 & 2 & 14 & 14 \\ \hline
        30 & 2: 10, 9, 3 & 2 & 3 & 3 \\ \hline
        31 & 2: 14, 4, 19 & 2 & 19 & 4 \\ \hline
        32 & 14: 2, 11, 10 & 2 & 14 & 11 \\ \hline
        33 & 7: 9, 14, 8 & 2 & 7 & 7 \\ \hline
        34 & 14: 2, 10, 6 & 2 & 14 & 10 \\ \hline
        35 & 2: 14, 15, 8 & 2 & 15 & 2 \\ \hline
        36 & 2: 14, 12, 16 & 2 & 16 & 16 \\ \hline
        37 & 8: 20, 3, 18 & 2 & 3 & 3 \\ \hline
        38 & 9: 14, 21, 4 & 2 & 9 & 9 \\ \hline
        39 & 2: 11, 7, 10, 14: 17, 12, 21 & 2 & 2 & 2 \\ \hline
        40 & 8: 9, 4, 20 & 3 & 8 & 8 \\ \hline
        41 & 2: 3, 6, 12. & 3 & 3 & 3 \\ \hline
        42 & 2: 3, 10, 21 & 3 & 21 & 21 \\ \hline
        43 & 14: 2, 12, 13 & 3 & 13 & 13 \\ \hline
        44 & 18: 15, 20, 13 & 3 & 18 & 18 \\ \hline
        45 & 21: 14, 10, 6 & 3 & 21 & 21 \\ \hline
        46 & 9: 13, 5, 8, 12, 17, 21 & 3 & 9 & 17 \\ \hline
        47 & 6: 12, 18, 8 & 3 & 6 & 12 \\ \hline
        48 & 12: 13, 2, 18 & 3 & 12 & 13 \\ \hline
        49 & 2: 3,10,14, & 3 & 3 & 3 \\ \hline
        50 & 8: 9, 10, 18 & 3 & 8 & 10 \\ \hline
        51 & 10: 8, 12, 14 & 3 & 10 & 12 \\ \hline
        52 &  & 4 & 16 & 1 \\ \hline
        53 & 4: 10, 2, 8 & 4 & 4 & 4 \\ \hline
        54 & 6: 12, 17, 9 & 4 & 17 & 17 \\ \hline
        55 & 1: 10, 3, 8 & 4 & 1 & 3 \\ \hline
        56 & 20: 15, 19, 16 & 4 & 20 & 20 \\ \hline
        57 & 10: 4, 2, 14 & 4 & 10 & 10 \\ \hline
        58 & 10: 4, 21, 9 & 4 & 10 & 10 \\ \hline
        59 & 6: 1, 14, 18 & 4 & 1 & 18 \\ \hline
        60 & 6: 2, 4, 8 & 4 & 6 & 6 \\ \hline
        61 & 4: 10, 21 & 4 & 4 & 4 \\ \hline
        62 & 19: 18, 9, 8 & 4 & 19 & 19 \\ \hline
        63 & 8: 4, 10 & 4 & 8 & 4 \\ \hline
        64 & 2: 14, 7, 6 & 4 & 2 & 14 \\ \hline
        65 &  & 6 & 16 & 3 \\ \hline
        66 &  & 6 & 16 & 5 \\ \hline
        67 & 6: 10, 21, 14 & 6 & 21 & 21 \\ \hline
        68 & 6: 5, 10, 7 & 6 & 6 & 6 \\ \hline
        69 & 6: 11, 9, 7 & 6 & 6 & 6 \\ \hline
        70 &  & 6 & 16 & 10 \\ \hline
        71 & 6: 8, 4, 10 & 6 & 6 & 4 \\ \hline
        72 & 6: 8, 7 & 6 & 6 & 7 \\ \hline
        73 & 20: 7, 6, 12 & 6 & 20 & 20 \\ \hline
        74 & 6: 20, 8 & 6 & 6 & 8 \\ \hline
        75 & 6: 9, 3, 7 & 6 & 3 & 7 \\ \hline
        76 & 6: 18, 9, 5 & 6 & 6 & 5 \\ \hline
        77 & 9: 20, 12, 6 & 6 & 9 & 9 \\ \hline
        78 &  & 7 & 17 & 19 \\ \hline
        79 & 14: 2, 20, 9 & 7 & 14 & 14 \\ \hline
        80 & 8: 6, 9 & 7 & 8 & 9 \\ \hline
        81 & 2: 12, 14, 8 & 7 & 2 & 2 \\ \hline
        82 & 2: 11, 6, 7 & 7 & 2 & 7 \\ \hline
        83 & 7: 2, 19, 14 & 7 & 7 & 7 \\ \hline
        84 & 15: 7, 18, 9 & 7 & 15 & 15 \\ \hline
        85 & 6: 7, 8, 3 & 7 & 3 & 6 \\ \hline
        86 & 8: 6, 9, 10, 20, 15, 12, 16 & 7 & 16 & 16 \\ \hline
        87 & 3: 6, 8, 5 & 7 & 3 & 3 \\ \hline
        88 & 12: 14 ,21, 2 & 7 & 12 & 12 \\ \hline
        89 & 8: 9, 6, 18 & 7 & 8 & 18 \\ \hline
        90 & 8: 6, 18, 7 & 8 & 7 & 7 \\ \hline
        91 & 1: 20, 6, 18 & 8 & 1 & 18 \\ \hline
        92 & 8: 9, 6, 12 & 8 & 8 & 12 \\ \hline
        93 & 8: 6, 20, 8 & 8 & 8 & 8 \\ \hline
        94 & 11 & 8 & 11 & 11 \\ \hline
        95 & 6: 8, 14, 2 & 8 & 6 & 8 \\ \hline
        96 & 8: 2, 14, 15 & 8 & 15 & 15 \\ \hline
        97 & 8: 7, 6, 10 & 8 & 7 & 8 \\ \hline
        98 & 8: 10, 11, 18 & 8 & 8 & 11 \\ \hline
        99 & 12: 9, 8, 14 & 8 & 12 & 12 \\ \hline
        100 & 18: 8, 15, 9 & 8 & 18 & 18 \\ \hline
        101 & 8: 10, 14, 4 & 8 & 8 & 4 \\ \hline
        102 & 8: 11, 6, 3 & 8 & 3 & 3 \\ \hline
        103 & 8: 7, 6, 9, 20 & 8 & 7 & 7 \\ \hline
        104 &  & 9 & 17 & 3 \\ \hline
        105 & 9: 18, 15, 16 & 9 & 9 & 16 \\ \hline
        106 & 8: 9, 20, 13 & 9 & 13 & 13 \\ \hline
        107 & 9: 6, 10, 8 & 9 & 9 & 8 \\ \hline
        108 & 9: 6, 8, 4 & 9 & 9 & 4 \\ \hline
        109 & 9: 8, 15, 10 & 9 & 9 & 9 \\ \hline
        110 & 9: 7, 14, 18 & 9 & 9 & 5 \\ \hline
        111 & 8: 18, 6, 9 & 9 & 8 & 9 \\ \hline
        112 & 9: 12, 20, 7 & 9 & 9 & 9 \\ \hline
        113 & 8: 9, 15, 18 & 9 & 15 & 8 \\ \hline
        114 & 11: 9, 2, 14 & 9 & 11 & 11 \\ \hline
        115 & 4: 9, 17, 18 & 9 & 4 & 4 \\ \hline
        116 & 9: 18, 16, 15 & 9 & 9 & 9 \\ \hline
        117 & 10: 4, 3, 7 & 10 & 10 & 7 \\ \hline
        118 & 12: 14, 21, 18 & 10 & 12 & 21 \\ \hline
        119 & 10: 15, 18, 3 & 10 & 10 & 10 \\ \hline
        120 & 8: 10, 6 & 10 & 8 & 6 \\ \hline
        121 & 10: 3, 1, 8 & 10 & 10 & 10 \\ \hline
        122 & 10: 21, 3, 1 & 10 & 10 & 21 \\ \hline
        123 & 19: 16, 9, 10 & 10 & 19 & 19 \\ \hline
        124 & 2: 10, 14, 20 & 10 & 2 & 20 \\ \hline
        125 & 10: 1, 21, 20 & 10 & 10 & 10 \\ \hline
        126 & 8: 18, 14, 16 & 10 & 16 & 8 \\ \hline
        127 & 10: 2, 11, 4 & 10 & 10 & 10 \\ \hline
        128 & 11: 2, 14, 20 & 11 & 11 & 14 \\ \hline
        129 & 11 & 11 & 11 & 11 \\ \hline
        130 & 11: 14, 20, 12 & 11 & 11 & 11 \\ \hline
        131 & 11: 10, 2, 17 & 11 & 17 & 17 \\ \hline
        132 & 11: 2, 14, 20 & 11 & 11 & 11 \\ \hline
        133 & 11 & 11 & 11 & 1 \\ \hline
        134 & 11: 2, 14, 12 & 11 & 11 & 12 \\ \hline
        135 & 11: 6, 1, 10, 19, 14, 15, 21 & 11 & 1 & 11 \\ \hline
        136 & 11: 14, 20, 9 & 11 & 11 & 20 \\ \hline
        137 & 2: 11, 14, 12 & 11 & 2 & 11 \\ \hline
        138 & 11: 2, 14, 18 & 11 & 18 & 3 \\ \hline
        139 & 11 & 11 & 11 & 11 \\ \hline
        140 & 11: 5, 2, 4. & 11 & 11 & 5 \\ \hline
        141 & 2: 11, 10, 7, 5, 14, 21, 18, 20 & 11 & 21 & 7 \\ \hline
        142 & 11: 12, 14, 8 & 11 & 11 & 8 \\ \hline
        143 & 11: 2, 7, 18 & 11 & 7 & 11 \\ \hline
        144 & 20: 16, 6, 8 & 12 & 20 & 20 \\ \hline
        145 & 2: 14, 13, 12 & 12 & 13 & 13 \\ \hline
        146 & 12: 6, 17, 10 & 12 & 17 & 17 \\ \hline
        147 & 12: 9, 20, 6 & 12 & 12 & 12 \\ \hline
        148 & 1: 3, 6, 10, 12, 16, 19, 21 & 12 & 1 & 16 \\ \hline
        149 & 2: 11, 14, 15 & 12 & 15 & 15 \\ \hline
        150 & 20: 8, 6, 13 & 12 & 20 & 20 \\ \hline
        151 & 14: 10, 21, 4 & 12 & 21 & 14 \\ \hline
        152 & 13: 2, 12, 20 & 12 & 13 & 13 \\ \hline
        153 & 12: 21, 18, 14 & 12 & 12 & 21 \\ \hline
        154 & 6: 9, 8 & 12 & 6 & 6 \\ \hline
        155 & 12: 10, 2, 20 & 12 & 12 & 12 \\ \hline
        156 & 20: 14, 8, 12 & 12 & 20 & 20 \\ \hline
        157 & 11: 12, 2, 14 & 12 & 11 & 11 \\ \hline
        158 & 12: 10, 3, 2 & 12 & 12 & 2 \\ \hline
        159 & 11: 2, 14, 13 & 12 & 13 & 11 \\ \hline
        160 & 9: 8, 6, 11 & 13 & 9 & 9 \\ \hline
        161 & 14: 12, 16, 21 & 13 & 16 & 16 \\ \hline
        162 & 10: 11, 3, 20, 4 & 13 & 10 & 10 \\ \hline
        163 & 11: 12, 14, 2 & 13 & 11 & 14 \\ \hline
        164 & 8: 20, 14, 9 & 13 & 8 & 20 \\ \hline
        165 & 12: 13, 20, 2 & 13 & 13 & 13 \\ \hline
        166 & 13: 4, 8, 17 & 13 & 13 & 13 \\ \hline
        167 & 17: 12, 14, 15 & 13 & 17 & 17 \\ \hline
        168 & 6: 10, 14, 8 & 13 & 6 & 14 \\ \hline
        169 & 4: 18, 8, 10 & 13 & 4 & 4 \\ \hline
        170 &  & 13 & 17 & 19 \\ \hline
        171 & 7: 4, 2, 20, 12, 14 & 13 & 7 & 7 \\ \hline
        172 & 14: 20, 18, 12 & 13 & 14 & 12 \\ \hline
        173 & 2: 6, 7, 9 & 13 & 2 & 7 \\ \hline
        174 &  & 13 & 19 & 21 \\ \hline
        175 &  & 14 & 19 & 22 \\ \hline
        176 & 11: 13, 14, 20 & 14 & 13 & 13 \\ \hline
        177 & 9: 21, 17, 14 & 14 & 9 & 17 \\ \hline
        178 & 12: 17, 21, 19 & 14 & 19 & 12 \\ \hline
        179 & 13: 12, 2, 9 & 14 & 13 & 13 \\ \hline
        180 & 2: 4, 10, 14 & 14 & 2 & 4 \\ \hline
        181 & 2: 14, 7, 8 & 14 & 2 & 7 \\ \hline
        182 & 6: 4, 8, 15 & 14 & 6 & 15 \\ \hline
        183 & 2: 12, 14, 4 & 14 & 2 & 4 \\ \hline
        184 & 14: 9, 2, 19 & 14 & 19 & 14 \\ \hline
        185 & 14: 8, 20, 2 & 14 & 14 & 2 \\ \hline
        186 & 2: 14, 17, 18, 20 & 14 & 17 & 17 \\ \hline
        187 & 12: 14, 8, 6 & 14 & 12 & 14 \\ \hline
        188 & 14: 18, 6, 8 & 14 & 14 & 8 \\ \hline
        189 & 2: 14, 15, 10 & 14 & 15 & 2 \\ \hline
        190 & 9: 6, 12, 15 & 15 & 9 & 9 \\ \hline
        191 & 10: 8, 6, 3 & 15 & 10 & 3 \\ \hline
        192 & 15: 18, 9 & 15 & 15 & 15 \\ \hline
        193 & 15: 10, 18, 21 & 15 & 15 & 15 \\ \hline
        194 & 17: 14, 15, 21 & 15 & 17 & 17 \\ \hline
        195 & 4: 10, 2, 8 & 15 & 4 & 4 \\ \hline
        196 & 15: 7, 8, 2 & 15 & 15 & 15 \\ \hline
        197 & 15: 8, 9, 6 & 15 & 15 & 15 \\ \hline
        198 & 16: 15, 20 & 15 & 16 & 16 \\ \hline
        199 & 12: 13, 20, 7 & 15 & 13 & 12 \\ \hline
        200 & 21: 15, 6, 14 & 15 & 21 & 21 \\ \hline
        201 & 3: 16, 15, 3 & 15 & 3 & 3 \\ \hline
        202 & 18: 15, 9, 8 & 15 & 18 & 18 \\ \hline
        203 & 15: 18, 8, 9 & 15 & 15 & 15 \\ \hline
        204 &  & 16 & 19 & 13 \\ \hline
        205 &  & 16 & 19 & 16 \\ \hline
        206 & 8: 7, 14, 20 & 16 & 7 & 14 \\ \hline
        207 & 6: 14, 17 & 16 & 17 & 17 \\ \hline
        208 & 19: 13, 14, 16 & 16 & 19 & 19 \\ \hline
        209 & 8: 20, 18 & 16 & 8 & 20 \\ \hline
        210 & 8: 15, 16, 20 & 16 & 16 & 8 \\ \hline
        211 & 8: 3, 6, 14 & 16 & 3 & 8 \\ \hline
        212 & 10: 4, 23, 14 & 16 & 10 & 4 \\ \hline
        213 & 2: 7, 9, 4 & 17 & 2 & 2 \\ \hline
        214 & 17: 14, 21, 18 & 17 & 17 & 17 \\ \hline
        215 & 9: 11, 2, 13 & 17 & 13 & 13 \\ \hline
        216 & 18: 15, 7, 8 & 17 & 18 & 15 \\ \hline
        217 & 16: 19, 13, 12 & 17 & 16 & 16 \\ \hline
        218 & 6: 14, 12, 3 & 17 & 3 & 3 \\ \hline
        219 & 18: 12, 15, 17 & 17 & 18 & 17 \\ \hline
        220 & 9: 8, 6, 7 & 17 & 9 & 9 \\ \hline
        221 & 17: 9, 8, 18 & 17 & 17 & 17 \\ \hline
        222 &  & 17 & 19 & 21 \\ \hline
        223 & 11: 18, 12, 9 & 17 & 18 & 9 \\ \hline
        224 & 20: 8, 6, 15 & 17 & 20 & 20 \\ \hline
        225 & 6: 9, 8, 18 & 17 & 6 & 9 \\ \hline
        226 & 9: 6, 13, 17 & 17 & 17 & 17 \\ \hline
        227 & 16: 21, 15 & 17 & 16 & 16 \\ \hline
        228 &  & 18 & 19 & 22 \\ \hline
        229 &  & 18 & 19 & 1 \\ \hline
        230 & 12: 9, 18, 4 & 18 & 12 & 12 \\ \hline
        231 & 8: 9, 18, 7 & 18 & 7 & 8 \\ \hline
        232 & 11: 2, 8, 14 & 18 & 11 & 2 \\ \hline
        233 & 18: 8, 15, 6 & 18 & 18 & 18 \\ \hline
        234 & 18: 15, 14, 8 & 18 & 18 & 18 \\ \hline
        235 & 10: 12, 5 & 18 & 10 & 10 \\ \hline
        236 & 18: 21, 14, 8 & 18 & 18 & 18 \\ \hline
        237 & 12: 20, 14, 2 & 18 & 12 & 2 \\ \hline
        238 & 18: 14, 20, 2 & 18 & 18 & 18 \\ \hline
        239 & 8: 9, 18, 15 & 18 & 15 & 15 \\ \hline
        240 & 8: 12, 14, 2 & 18 & 8 & 8 \\ \hline
        241 & 2: 8, 14, 18 & 18 & 2 & 2 \\ \hline
        242 & 18: 15, 7, 8 & 18 & 18 & 18 \\ \hline
        243 & 13: 12, 20, 14 & 18 & 13 & 13 \\ \hline
        244 & 20:12 & 19 & 20 & 20 \\ \hline
        245 & 2: 10, 6 & 19 & 2 & 10 \\ \hline
        246 & 2: 12, 14, 16 & 19 & 16 & 16 \\ \hline
        247 & 14: 19, 21, 10 & 19 & 19 & 21 \\ \hline
        248 & 6: 10, 12, 16 & 19 & 16 & 6 \\ \hline
        249 & 2: 8, 21, 17 & 19 & 21 & 17 \\ \hline
        250 & 14: 19, 15, 20 & 19 & 19 & 14 \\ \hline
        251 & 6: 8, 9, 15, 12, 18 & 19 & 15 & 15 \\ \hline
        252 & 14; 11, 4, 2 & 19 & 14 & 4 \\ \hline
        253 & 8: 7, 9, 18 & 19 & 7 & 18 \\ \hline
        254 & 10: 6, 4, 18, 15, 20, 19 & 19 & 10 & 15 \\ \hline
        255 & 8: 14, 2, 7 & 19 & 7 & 8 \\ \hline
        256 & 19: 16, 12, 18 & 19 & 19 & 19 \\ \hline
        257 & 9: 20, 15, 6 & 19 & 9 & 9 \\ \hline
        258 & 3: 2, 20, 19 & 19 & 3 & 3 \\ \hline
        259 & 9: 2, 18, 13 & 20 & 13 & 13 \\ \hline
        260 & 6: 12, 20, 7 & 20 & 7 & 12 \\ \hline
        261 &  & 20 & 21 & 15 \\ \hline
        262 & 14: 10, 8, 18 & 20 & 14 & 18 \\ \hline
        263 & 20: 21, 18, 17 & 20 & 20 & 20 \\ \hline
        264 & 7: 20, 9, 17 & 20 & 7 & 7 \\ \hline
        265 & 20: 6, 14, 2 & 20 & 20 & 6 \\ \hline
        266 & 9: 2, 17 & 20 & 17 & 9 \\ \hline
        267 & 8: 7, 20, 6 & 20 & 8 & 6 \\ \hline
        268 & 20: 9, 8, 15 & 20 & 20 & 20 \\ \hline
        269 & 6: 8, 2, 10, 13, 12, 20, 18 & 20 & 13 & 18 \\ \hline
        270 & 14: 2, 8, 11 & 20 & 14 & 11 \\ \hline
        271 & 18: 14, 12, 9 & 20 & 18 & 18 \\ \hline
        272 & 14: 12, 20, 15 & 20 & 15 & 14 \\ \hline
        273 & 21: 17, 18, 10 & 21 & 21 & 21 \\ \hline
        274 & 8: 20, 21, 14 & 21 & 21 & 21 \\ \hline
        275 & 14: 2, 6, 4 & 21 & 14 & 14 \\ \hline
        276 & 12: 14, 13, 20 & 21 & 13 & 12 \\ \hline
        277 & 8; 6, 7, 20 & 21 & 20 & 6 \\ \hline
        278 & 12: 14, 15, 2 & 21 & 12 & 15 \\ \hline
        279 & 10: 3, 21 & 21 & 10 & 10 \\ \hline
        280 & 10; 11, 12 ,21 & 21 & 21 & 10 \\ \hline
        281 &  & 21 & 21 & 10 \\ \hline
        282 & 9: 20, 8, 6 & 21 & 9 & 6 \\ \hline
        283 & 21: 14,1,10 & 21 & 21 & 21 \\ \hline
        284 & 2: 6, 7, 21, 17, 14, 12 & 21 & 21 & 2 \\ \hline
        285 & 17: 13, 21,8 & 21 & 17 & 17 \\ \hline
        286 & 10: 21, 15 & 21 & 21 & 21 \\ \hline
        287 & 14: 3, 11 & 21 & 3 & 11 \\ \hline
        288 &  & 82-188/S14 & 21 & 16 \\ \hline
        289 & 14: 2, 8, 4 & 82-188/S14 & 14 & 4 \\ \hline
        290 & 7: 2, 8, 10 & 82-188/S14 & 7 & 7 \\ \hline
        291 & 10: 14, 8 & 82-188/S14 & 10 & 8 \\ \hline
        292 & 2: 7, 8, 4 & 82-188/S14 & 7 & 2 \\ \hline
        293 & 2: 4, 7, 9 & 82-188/S14 & 7 & 2 \\ \hline
        294 & 12: 20, 19 & 82-188/S14 & 20 & 19 \\ \hline
        295 & 6: 8, 14, 10 & 82-188/S14 & 6 & 6 \\ \hline
        296 & 20: 12, 13, 14 & 82-188/S14 & 20 & 13 \\ \hline
        297 & 2: 11, 12, 14 & 82-188/S14 & 2 & 11 \\ \hline
        298 & 14: 2, 8, 3 & 82-188/S14 & 14 & 14 \\ \hline
        299 & 12: 14, 2, 10 & 82-188/S14 & 12 & 12 \\ \hline
        300 & 9: 7, 20, 10 & 82-188/S14 & 20 & 7 \\ \hline
        301 & 11: 4, 2, 10 & 82-188/S14 & 11 & 4 \\ \hline
        302 & 12: 20, 9, 8 & 82-188/S14 & 20 & 9 \\ \hline
        303 & 6: 8, 10, 14 & 82-188/S14 & 6 & 6 \\ \hline
        304 & 6: 7, 8, 15 & 82-188/S14 & 15 & 6 \\ \hline
        305 & 6: 8, 9, 10 & 85-131/S14 & 6 & 6 \\ \hline
        306 & 9: 2, 6,18 & 85-131/S14 & 18 & 2 \\ \hline
        307 & 6: 8,9,18 & 85-131/S14 & 18 & 9 \\ \hline
        308 & 8: 6, 9, 10 & 85-131/S14 & 8 & 6 \\ \hline
\end{longtable}

\vspace{\fill}
\pagebreak
\subsection*{Appendix B: Source Code for Exact Algorithm with Hungarian}

\par Dependencies:
\begin{itemize}
    \item Python \texttt{numpy} package: available at \url{http://www.numpy.org/};
    \item Python \texttt{Munkres} package: available at \url{https://pypi.python.org/pypi/munkres/};
\end{itemize}

\definecolor{mygreen}{rgb}{0,0.6,0}
\definecolor{mygray}{rgb}{0.5,0.5,0.5}
\definecolor{mymauve}{rgb}{0.58,0,0.82}

\lstset{ %
  backgroundcolor=\color{white},   % choose the background color
  basicstyle=\footnotesize,        % size of fonts used for the code
  breaklines=true,                 % automatic line breaking only at whitespace
  captionpos=b,                    % sets the caption-position to bottom
  commentstyle=\color{mygreen},    % comment style
  escapeinside={\%*}{*)},          % if you want to add LaTeX within your code
  keywordstyle=\color{blue},       % keyword style
  stringstyle=\color{mymauve},     % string literal style
}

\begin{lstlisting}[language=python]
from munkres import Munkres, print_matrix
import numpy, sys
import re

def getCost(tier):
    return 2 * tier * tier

# Given a column number for augmented matrix, find its seminar number.
def getSeminar(index, q):
    s = 0
    for i in xrange(len(q)):
        s += q[i]
        if index < s:
            return i + 1
    return -1   # Shouldn't happen!

# Parse an input line to (seminar, tier) pairs
def parseLine(line):
    tierStrings = re.split(':|;', line)
    splitTier = [x.split(",") for x in tierStrings]
    splitTierResult = list()
    for tier in xrange(len(splitTier)):
        splitTierResult.append(list())
        for selection in splitTier[tier]:
            sanitizedSelection = re.sub("\D", "", selection)
            if re.sub("\D", "", sanitizedSelection) != "":
                splitTierResult[tier].append(int(sanitizedSelection))
    result = dict()
    for i in xrange(len(splitTierResult)):
        for s in splitTierResult[i]:
            if not(1 <= s and s <= m):
                print "Warning: Student entered out-of-bound Seminar ID: %s" % line
            elif s in result:
                print "Warning: Student has multiple entry for %d on line %s" % (s, line)
            else:
                result[s] = getCost(i)

    return result

#
# !-- Entry point --!
#
# Specify input file name. 
inputPath = "data/Fall2014"

# Ask for parameters from user
m = int(raw_input("Enter number of seminars (Seminar ID starts from 1): "))
qq = raw_input("Enter quotas for %d seminars: " % m)
if len(qq.split()) == 1:
    q = [int(qq.split()[0]) for i in xrange(m)]
elif len(qq.split()) == m:
    q = [int(qq.split()[i]) for i in xrange(m)]
else:
    print "Invalid quota input!"
    sys.exit(1)

# Load student selections from input file.
print "Parsing input file `%s`..." % inputPath
with open(inputPath, 'r') as f:
    userInput = f.readlines()
if userInput[-1].startswith("END"):
    userInput = userInput[:-1]
else:
    raise Exception("Last line of file must be END!")
    sys.exit(1)
A = [parseLine(line) for line in userInput] # Parse input lines
n = len(A)
print "Number of students: %s" % n
if sum(q) < n:
    print "Quota cannot fit all students!"
    sys.exit(1)

# Transfer array to Student-Seminar.
B = []
MCOST = 100000
for i in xrange(n):
    B.append([A[i][j] if j in A[i] else MCOST for j in xrange(1, m+1)])
# Duplicate columns for hungarian
B = numpy.array(B, dtype='int32')
Slices = list()
for i in xrange(m):
    Slices.append(numpy.tile(numpy.transpose([B[:,i]]), q[i]))
C = numpy.concatenate(tuple(Slices), axis=1)
# Add zero rows for dummy students
for i in xrange(sum(q) - n):
    C = numpy.vstack([C, numpy.zeros(sum(q), dtype='int32')])
C = C.astype(int)

# Apply Munkres library to calculate Hungarian.
print "Running Hungarian algorithm on matrix of dimension", C.shape
C = C.tolist()
m = Munkres()
indexes = m.compute(C)
total = 0
print "Student ID, Assigned Seminar, Cost"
for row, column in indexes:
    if row >= n: continue # Skip dummy students
    value = C[row][column]
    total += value
    print '%d, %d, %d' % (row, getSeminar(column, q), value)
print 'Total Cost: %d' % total
\end{lstlisting}

\vspace{\fill}
\pagebreak
\subsection*{Appendix C: Source Code for Randomized Approximation Algorithm}
\par Dependencies:
\begin{itemize}
    \item Python \texttt{numpy} package: available at \url{http://www.numpy.org/};
\end{itemize}

\begin{lstlisting}[language=python]
import math, numpy, sys
import re
import copy
import random

def getCost(tier):
    return 2 * tier * tier

# Parse an input line to (seminar, tier) pairs
def parseLine(line):
    tierStrings = re.split(':|;', line)
    splitTier = [x.split(",") for x in tierStrings]
    splitTierResult = list()
    for tier in xrange(len(splitTier)):
        splitTierResult.append(list())
        for selection in splitTier[tier]:
            sanitizedSelection = re.sub("\D", "", selection)
            if re.sub("\D", "", sanitizedSelection) != "":
                splitTierResult[tier].append(int(sanitizedSelection))
    
    result = dict()
    for i in xrange(len(splitTierResult)):
        for s in splitTierResult[i]:
            if not(1 <= s and s <= m):
                print "Warning: Student entered out-of-bound Seminar ID: %s" % line
            elif s in result:
                print "Warning: Student has multiple entry for %d on line %s" % (s, line)
            else:
                result[s] = getCost(i)

    return result

# Main function for running randomized assignment.
def randomAsgn(quotaRatio):
    roster = [list() for i in xrange(m+1)]
    currCost = 0
    currAsgn = dict()
    availStudents = set(range(n))
    # Stage One: Assign first choices, up to set quota.
    for s in xrange(1, m+1):
        tmp = randomSub(set(R[s][0]) & availStudents,
                int(math.floor(float(q[s-1]) * quotaRatio)))
        roster[s] += tmp
        currCost += getCost(0) * len(tmp)
        for student in tmp:
            currAsgn[student] = s
        availStudents = availStudents.difference(set(tmp))
    # Stage Two: Assign second choices, as many as possible.
    for s in randomly(xrange(1, m+1)):
        tmp = randomSub(set(R[s][1]) & availStudents, q[s-1] - len(roster[s]))
        roster[s] += tmp
        currCost += getCost(1) * len(tmp)
        for student in tmp:
            currAsgn[student] = s
        availStudents = availStudents.difference(set(tmp))
    # Stage Three: Enroll remanining students in Round-robin fashion.
    s = 1
    while len(availStudents) > 0:
        if len(roster[s]) < q[s-1]:
            student = availStudents.pop()
            roster[s].append(student)
            currCost += MCOST
            currAsgn[student] = s
        s = 1 if s == m else s+1    
    return (currCost, currAsgn, roster)

# Getting random sublist of a set with length l
def randomSub(L, l):
    return random.sample(list(L), min(len(list(L)), l))

# Gives random iterator of a list
def randomly(seq):
    shuffled = list(seq)
    random.shuffle(shuffled)
    return iter(shuffled)


#
# !-- Entry point --!
#
# Specify input file name. 
inputPath = "data/Fall2014"

# Ask for parameters from user
m = int(raw_input("Enter number of seminars (Seminar ID starts from 1): "))
qq = raw_input("Enter quotas for %d seminars: " % m)
if len(qq.split()) == 1:
    q = [int(qq.split()[0]) for i in xrange(m)]
elif len(qq.split()) == m:
    q = [int(qq.split()[i]) for i in xrange(m)]
else:
    print "Invalid quota input!"
    sys.exit(1)
iters = int(raw_input("Enter number of iterations to run randomized assignment: "))

# Load student selections from input file.
print "Parsing input file `%s`..." % inputPath
with open(inputPath, 'r') as f:
    userInput = f.readlines()
if userInput[-1].startswith("END"):
    userInput = userInput[:-1]
else:
    raise Exception("Last line of file must be END!")
    sys.exit(1)
A = [parseLine(line) for line in userInput] # Parse input lines
n = len(A)
print "Number of students: %s" % n
if sum(q) < n:
    print "Quota cannot fit all students!"
    sys.exit(1)

# Transfer input to (student,seminar) -> ranking mapping.
# Initialize (seminar, ranking) -> student mapping.
B = []
R = [([], [])]
MCOST = 100000
for i in xrange(n):
    B.append([A[i][j] if j in A[i] else MCOST for j in xrange(1, m+1)])
for i in xrange(1, m+1):
    rankedFirst = list()
    rankedSecond = list()
    for s in xrange(n):
        if i in A[s] and A[s][i] == getCost(0):
            rankedFirst.append(s)
        elif i in A[s] and A[s][i] == getCost(1):
            rankedSecond.append(s)
    R.append((rankedFirst, rankedSecond))

# Mainloop for iterations
bestCost = sys.maxint
bestAsgn = list()
roster = list()
bestRatio = 0.0
for i in xrange(iters):
    for j in xrange(1, max(q)+1):
        (currCost, currAsgn, currRoster) = randomAsgn(j * 1.0 / max(q))
        if currCost < bestCost:
            print "Current optimal cost: %d... [Run %d out of %d]" % (currCost, i+1, iters)
            bestCost = currCost
            bestAsgn = copy.deepcopy(currAsgn)
            bestRoster = copy.deepcopy(currRoster)
            bestRatio = j * 1.0 / max(q)

print "Best Cost: %d  (with first-choice quota %f)" % (bestCost, bestRatio)
print "Assignment as follows:"
for i in xrange(n):
    print "%d -> %d" % (i, bestAsgn[i])
print "--------------------------------"
print "Seminar roster:"
for i in xrange(1, m+1):
    print "Seminar %d:" % i, bestRoster[i]
\end{lstlisting}

\vspace{\fill}
\end{document}
