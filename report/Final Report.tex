\documentclass{article} % For LaTeX2e
\usepackage{nips13submit_e,times}
%\usepackage{hyperref}
\usepackage[hyphens]{url}
\usepackage{bbm}
\usepackage{amsfonts}
\usepackage{alltt}
\usepackage{multirow, caption}
\usepackage{algorithm}
\usepackage{amsmath}
\usepackage{graphicx}
\usepackage[noend]{algpseudocode}
\usepackage{etoolbox}
\usepackage{hyperref}
\usepackage{fancyvrb}
\usepackage{tgcursor}
\newcommand\userinput[1]{\textbf{#1}}
\newcommand\comment[1]{\textit{#1}}
\newcommand\stdout[1]{\textsl{#1}}

\makeatletter
\preto{\@verbatim}{\topsep=0pt \partopsep=0pt }
\makeatother

% Tweak algorithmic package.
\renewcommand{\algorithmicrequire}{\textbf{Input:}}
\renewcommand{\algorithmicensure}{\textbf{Output:}}

\title{Operations Research II\\Final Report}

\author{
Keenan Gao \\
\And
Binghui Ouyang \\
\And
Hanwen Zhang \\
\And
Yiming Zong\\ \\
\and Department of Mathematical Sciences \\ Carnegie Mellon University \\ Pittsburgh, PA 15213
}

\newcommand{\fix}{\marginpar{FIX}}
\newcommand{\new}{\marginpar{NEW}}

\nipsfinalcopy % Uncomment for camera-ready version

\begin{document}

% Title
\maketitle

% Paper Abstract
\begin{abstract}
    Space reserved for abstract.
\end{abstract}

%
% Main Body Starts
%
\section{Problems Overview}
    \texttt{TODO: Overview of the problem in words.}


\section{Mathematical Model}
\subsection{General Input}
\begin{itemize}
    \item $n$: Number of students ($n > 0$);
    \item $m$: Number of seminars ($m > 0$);
    \item $k$: Max number of selections that a student can make ($1 \leq k \leq m$);
    \item $s_{i,j}$: The $j^{\text{th}}$ selection of $i^{\text{th}}$ student, where $1\leq i \leq n$ and $1\leq j \leq k$. $s_{i,j}=0$ when the Student $i$ makes no corresponding choice for Rank $j$;
    \item $q_k$: The quota for $k^{\text{th}}$ seminar, where $1\leq k \leq m$.
\end{itemize}

\subsection{Input Constraints}
\begin{itemize}
    \item Positivity: $n,m,k>0$, $\forall k\in\{1,\cdots,m\}, q_k>0$;
    \item Number of selections for student is bounded by number of available seminars: $k \leq m$;
    \item \textbf{(?)} Student rankings are valid and unique: $\forall (i,j), 1\leq s_{i,j} \leq m$. And, for each $i$, all non-zero entries $s_{i,j}$'s take unique values.
\end{itemize}

\subsection{Decision Variables}
\begin{itemize}
    \item $Y_{i,j}$: Indicator variables for whether Student $i$ is assigned to Seminar $j$, where $1\leq i \leq n$ and $1\leq j \leq k$;
\end{itemize}

\subsection{Data Pre-Processing}
    \par\qquad In order to deal with cases when a student is only willing or allowed to rank $k'<m$ seminars, we automatically set all ``unassigned'' priorities to $(k+1)$. Also, we change the representation of students' preference from \emph{(student, ranking) $\mapsto$ seminar} to \emph{(student, seminar) $\mapsto$ ranking} to make further calculations easier, i.e.
        $$X_{i,j}=\begin{cases}
                    l &\text{If Student $i$ ranked $j$ as $l^{\text{th}}$ option, or $s_{i,l}=j$ for some $l\in\{1,\cdots,k\}$}\\
                    {\bf\it M}     &\text{If Seminar $k$ is not on Student $i$'s list, or $s_{i,l} \neq j$ for all $l\in\{1,\cdots,k\}$}
                  \end{cases},$$
    where ${\bf\it M}$ is an arbitrarily large value in order to discourage the algorithm from assigning a student to a seminar that s/he did not list.

\subsection{General Constraints}
    \begin{itemize}
    \item $Y_{i,j}$'s are indeed indicator variables: $\forall (i,j), Y_{i,j}\in\mathbb{Z}, Y_{i,j}\geq 0, Y_{i,j}\leq 1$;
    \item Each student is assigned precisely one seminar: $\forall i$, $\sum_{l=1}^{m}{Y_{i,l}}=1$;
    \item Each seminar is within enrollment quota: $\forall j$, $\sum_{l=1}^{n}{Y_{l,j}} \leq q_j$;
\end{itemize}

%
% Heuristic Selection
%
\section{Approach for Various Heuristic Functions}
    \par\qquad Due to the flexibility of the original problem, we are proposing different objective functions for optimization, including minimizing the total ``rank'' given by the students, maximizing the number of student getting their top $\lambda k$ choice (where $\lambda\in(0,1)$), etc. In the following sub-sections we present our approach for each heuristic in mathematical terms.

\subsection{Minimize Total Rank of Students}
    \par\qquad In this case, our goal is to minimize the sum of all student rankings for their assigned seminars. To do so, our objective is to minimize $W = \sum_{i=1}^{n}{\sum_{j=1}^{m}{X_{i,j}Y_{i,j}}}.$
\subsection{Maximize Number of Students Getting Top-Tier Seminars}
    \par\qquad In this case, we would additionally require the user to input a value $\lambda\in(0,1)$, representing the range of rankings we consider as ``top-tier'', i.e. $\{1,\cdots,\lfloor \lambda k\rfloor\}$. Given this heuristic, our objective is to maximize $W = \sum_{i=1}^{n}{\sum_{j=1}^{m}{\mathbbm{1}_{X_{i,j} \leq \lfloor \lambda k\rfloor}Y_{i,j}}}.$

%
% Optimization Algorithm
%
\section{Approximation Algorithms}
    \par\qquad Given the constraints, our problem can be classified as a \emph{Genrealized Assignment Problem}. According to Martello and Toth\cite{mt90}, it is \emph{NP-hard}, so an approximation algorithm must be applied in order to solve the problem in a reasonable amount of time. Analogous to the \emph{Knapsack Problem}, our fundamental approach is the greedy algorithm (with variations), and then make finishing touch based on the principle of \emph{Stable Marriage Problem}. Following sub-sections will present the alrogithm in details:

\subsection{Ranking-Based Greedy Algorithm}
    \par For the greedy algorithm, we first satisfy (a portion of) all students' first choices, then second choices, and so on. Depending on the ``popularity'' of each seminar, we may limit the number of students allowed to be added to a seminar at each ranking. The algorithm (as \emph{Algorithm 1} on next page) is outlined as follows, and it can be run multiple times in order to select an assignment with least amount of students that are not assigned to their ranked list.
    \begin{algorithm}
        \caption{Ranking-Based Greedy Algorithm}
        \begin{algorithmic}
            \Ensure{$\text{asgn}_i \gets \text{seminar assignment for Student $i$ based on greedy algorithm}$}
            \For{r = 1 to k}    
                \For{i = 1 to m}
                    \State pool[i] $\gets \{\text{unassigned student s} \mid \text{s listed seminar $i$ as $r^\text{th}$ choice}\}$
                    \State pool[i] $\gets \text{random subset of itself with certain size limit (e.g. seminar quota)}$
                \EndFor
                \State Merge each pool[i] into asgn
            \EndFor
            \State Fill in still unassigned students
        \end{algorithmic}
    \end{algorithm}

\subsection{Stable Assignment Optimization}
    \par Similar to the principle of \emph{Stable Marriage Problem}, in our final seminar assignment we do not want to have two students \emph{A} and \emph{B}, such that \emph{A} prefers \emph{B}'s section, and also vice versa (we call those two students \emph{rogue pair}). This can be done by scanning each pair of students and fixing every \emph{rogue pair}. The algorithm is outlined in \emph{Algorithm 2} on the next page.
    \begin{algorithm}
        \caption{Rogue-Pair Fixing Algorithm}
        \begin{algorithmic}
        \Require $\text{asgn}_i \gets \text{current seminar assignment for Student $i$}$
        \Ensure (i,j) if we found a rogue pair, otherwise null
        \Function{FindRoguePair}{\text{asgn}}
            \For {i=1 to n}
                \For {j=i+1 to n}
                    \If{Student $i$ and $j$ prefer each other's seminar}
                        \Return{(i,j)}
                    \EndIf
                \EndFor
            \EndFor
            \Return{null}
        \EndFunction
        
        \null
        
        \Require $\text{asgn}_i \gets \text{seminar assignment for Student $i$ based on greedy algorithm}$
        \Ensure $\text{asgn}_i$: rogue pair-free assignment for Student $i$
                \State p $\gets$ FindRoguePair(asgn)
                \While {p {\bf not} null}
                    \State (i,j) $\gets$ p
                    \State $\text{asgn}_i \leftrightarrow \text{asgn}_j$
                    \State p $\gets$ FindRoguePair(asgn)
                \EndWhile
        
        \end{algorithmic}
    \end{algorithm}

\section{Exact Algorithm}
    \par\qquad The problem can also be reduced to an \emph{Assignment Problem} when we include ``dummy seminars'' and ``dummy students''. When we apply \emph{Hungarian Algorithm}, we can obtain an absolute optimal solution that minimizes the total rank of students. Details about the algorithm are as follows:
    
\subsection{Data Pre-Processing}
    \par\qquad We start with the matrix $X_{i,j}$ as obtained in \emph{Section 2.4}. For each column that represents Seminar $j$, we create extra $(j-1)$ dummy seminars by duplicating the same column $q_j$ times. After that, we make our cost matrix square by adding zero rows at the bottom of the cost matrix. This gives us a matrix that we may feed into \emph{Hungarian Algorithm}.

\subsection{Hungarian Algorithm}
    \par\qquad Given the square cost matrix from previous section, we may simply apply \emph{Hungarian Algorithm}, which returns a \emph{student-seminar} assignment with minimal total cost. The algorithm completes in polynomial time \texttt{[reference]}, and given the result we may simply assign each student to the actual seminar that the assignment corresponds to.
%
% Summary
%
\section{Summary of Results}
    \par The algorithms are tested on the real data for the incoming class of Dietrich College for Year 2013 ($n=309$). We have found that the approximation algorithm (...), while the \emph{Hungarian Algorithm} gives really satisfactory assignment within five minutes. Despite the satisfactory performance of the exact algorithm, its run-time complexity $\mathcal{O}(n^3)$ makes the algorithm undesirable for $n>1000$. Following is a comparison between the performance of manual assignment, approximation algorithm, and exact algorithm:

\section{Further Work \& Enhancements}

\subsection{Supporting bi-directional preference/cost parameters with Stable Marriage Algorithm}

\subsection{And more...}

\section{Acknowledgements}
Professor Frieze (weekly meeting, progress tracking)\\
Academic Advisor \& Associate Dean from Dietrich (meeting about requirements, real data for Year 2013; names?)\\
Brian Clapper (Python implementation of Hungarian)\\
More?
\urlstyle{rm}
\vskip .2in
\begin{thebibliography}{9}
\bibitem{mt90}
Martello, Silvano, and Paolo Toth. Knapsack Problems: Algorithms and Computer Implementations. Chichester: J. Wiley \& Sons, 1990. Print.
%\url{http://www.or.deis.unibo.it/kp/Chapter7.pdf}
\end{thebibliography}

\end{document}
