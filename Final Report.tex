\documentclass{article} % For LaTeX2e
\usepackage{nips13submit_e,times}
%\usepackage{hyperref}
\usepackage[hyphens]{url}
\usepackage{bbm}
\usepackage{amsfonts}
\usepackage{alltt}
\usepackage{multirow, caption}
\usepackage{algorithm}
\usepackage{amsmath}
\usepackage{graphicx}
\usepackage[noend]{algpseudocode}
\usepackage{etoolbox}
\usepackage{hyperref}
\usepackage{fancyvrb}
\usepackage{tgcursor}
\newcommand\userinput[1]{\textbf{#1}}
\newcommand\comment[1]{\textit{#1}}
\newcommand\stdout[1]{\textsl{#1}}

\makeatletter
\preto{\@verbatim}{\topsep=0pt \partopsep=0pt }
\makeatother

\title{Operations Research II\\Final Report}

\author{
Keenan Gao \\
\And
Binghui Ouyang \\
\And
Emmie Zhang \\
\And
Yiming Zong\\ \\
\and Department of Mathematical Sciences \\ Carnegie Mellon University \\ Pittsburgh, PA 15213
}

\newcommand{\fix}{\marginpar{FIX}}
\newcommand{\new}{\marginpar{NEW}}

\nipsfinalcopy % Uncomment for camera-ready version

\begin{document}

% Title
\maketitle

% Paper Abstract
\begin{abstract}
    Space reserved for abstract.
\end{abstract}

% Main Body Starts

\section{Problems Overview}
    \texttt{TODO: Overview of the problem in words.}


\section{Mathematical Model}
\subsection{General Input}
\begin{itemize}
    \item $n$: Number of students ($n > 0$);
    \item $m$: Number of seminars ($m > 0$);
    \item $k$: Max number of selections that a student can make ($1 \leq k \leq m$);
    \item $s_{i,j}$: The $j^{\text{th}}$ selection of $i^{\text{th}}$ student, where $1\leq i \leq n$ and $1\leq j \leq k$. $s_{i,j}=0$ when the Student $i$ makes no corresponding choice for Rank $j$;
    \item $q_k$: The quota for $k^{\text{th}}$ seminar, where $1\leq k \leq m$.
\end{itemize}

\subsection{Input Constraints}
\begin{itemize}
    \item Positivity: $n,m,k>0$, $\forall k\in\{1,\cdots,m\}, q_k>0$;
    \item Number of selections for student is bounded by number of available seminars: $k \leq m$;
    \item \textbf{(?)} Student rankings are valid and unique: $\forall (i,j), 1\leq s_{i,j} \leq m$. And, for each $i$, all non-zero entries $s_{i,j}$'s take unique values.
\end{itemize}

\subsection{Decision Variables}
\begin{itemize}
    \item $Y_{i,j}$: Indicator variables for whether Student $i$ is assigned to Seminar $j$, where $1\leq i \leq n$ and $1\leq j \leq k$;
\end{itemize}

\subsection{Data Pre-Processing}
    \par\qquad In order to deal with cases when a student is only willing or allowed to rank $k'<m$ seminars, we automatically set all ``unassigned'' priorities to $(k+1)$. Also, we change the representation of students' preference from \emph{(student, ranking) $\mapsto$ seminar} to \emph{(student, seminar) $\mapsto$ ranking} to make further calculations easier, i.e.
        $$X_{i,j}=\begin{cases}
                    l &\text{If Student $i$ ranked $j$ as $l^{\text{th}}$ option, or $s_{i,l}=j$ for some $l\in\{1,\cdots,k\}$}\\
                    k+1     &\text{If Seminar $k$ is not on Student $i$'s list, or $s_{i,l} \neq j$ for all $l\in\{1,\cdots,k\}$}
                  \end{cases}.$$

\subsection{General Constraints}
    \begin{itemize}
    \item $Y_{i,j}$'s are indeed indicator variables: $\forall (i,j), Y_{i,j}\in\mathbb{Z}, Y_{i,j}\geq 0, Y_{i,j}\leq 1$;
    \item Each student is assigned precisely one seminar: $\forall i$, $\sum_{l=1}^{m}{Y_{i,l}}=1$;
    \item Each seminar is within enrollment quota: $\forall j$, $\sum_{l=1}^{n}{Y_{l,j}} \leq q_j$;
\end{itemize}
\section{Approach for Various Heuristic Functions}
    \par\qquad Due to the flexibility of the original problem, we are proposing different objective functions for optimization, including minimizing the total ``rank'' given by the students, maximizing the number of student getting their top $\lambda k$ choice (where $\lambda\in(0,1)$), etc. In the following sub-sections we present our approach for each heuristic in mathematical terms.

\subsection{Minimize Total Rank of Students}
    \par\qquad In this case, our goal is to minimize the sum of all student rankings for their assigned seminars. To do so, our objective is to minimize $W = \sum_{i=1}^{n}{\sum_{j=1}^{m}{X_{i,j}Y_{i,j}}}.$
\subsection{Maximize Number of Students Getting Top-Tier Seminars}
    \par\qquad In this case, we would additionally require the user to input a value $\lambda\in(0,1)$, representing the range of rankings we consider as ``top-tier'', i.e. $\{1,\cdots,\lfloor \lambda k\rfloor\}$. Given this heuristic, our objective is to maximize $W = \sum_{i=1}^{n}{\sum_{j=1}^{m}{\mathbbm{1}_{X_{i,j} \leq \lfloor \lambda k\rfloor}Y_{i,j}}}.$

\subsection{And Potentially More}

\section{Summary of Results}

\section{Further Work \& Enhancements}

\section{References}
\urlstyle{rm}

\end{document}
